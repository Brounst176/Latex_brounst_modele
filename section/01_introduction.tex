\section{Introduction}
Ce document sert à reproduire un modèle Latex défini et une méthodologie pour créer des rapports Latex.
\section{Fonction mathématique}

\begin{equation}
    \boxed{m = F - \frac{k^TFR(M-S)}{k^T R(M-S)}}\\
    \label{colin}
\end{equation}
\begin{align*}
    x_m&=x_c+f\cdot\frac{r_{11}(X_M-X_S)+r_{21}(Y_M-Y_S)+r_{31}(Z_M-Z_S)}{r_{13}(X_M-X_S)+r_{23}(Y_M-Y_S)+r_{33}(Z_M-Z_S)}\\
    y_m&=y_c+f\cdot\frac{r_{12}(X_M-X_S)+r_{22}(Y_M-Y_S)+r_{32}(Z_M-Z_S)}{r_{13}(X_M-X_S)+r_{23}(Y_M-Y_S)+r_{33}(Z_M-Z_S)}\\
    z_m&=0
\end{align*}

\begin{align*}
    \descrVarEq{\scu^{cam}=(o,x,y,z)}{Système de coordonnée de la caméra}\\
    \descrVarEq{x,y}{Centre du capteur de l'image}\\
    \descrVarEq{z}{Axe de la visée de l'appareil photo}\\
    \descrVarEq{\scu^{glob}=(O,X,Y,Z)}{Système de coordonnées globale}\\
    \descrVarEq{m=[x_m,y_m,0]}{Coordonnée d'un point terrain sur le capteur, z=0 dans $\scu^{cam}$}\\
    \descrVarEq{M=[X_M,Y_M,Z_M]}{Coordonnée d'un point terrain dans $\scu^{glob}$}\\
    \descrVarEq{S=[X_S,Y_S,Z_S]}{Coordonnée du point de la caméra dans $\scu^{glob}$}\\
    R_{glob}^{cam}=R_\omega\cdot R_\phi\cdot R_\kappa=\begin{bmatrix}
    r_{11}&r_{21}&r_{31}\\
    r_{12}&r_{22}&r_{32}\\
    r_{13}&r_{23}&r_{33}
    \end{bmatrix}\textrm{ : }&\textrm{Matrice de rotation du $\scu^{glob}$vers $\scu^{cam}$}\\
    \descrVarEq{F=[x_c,y_c,-f]}{Point S dans $\scu^{cam}$}\\
    &\textrm{(Distance focale $f$)}\\
    &\textrm{($x_c,y_c$ sont égales à 0 dans un modèle théorique)}\\
\end{align*}
\begin{equation}
    \label{eq:transdlt}
    L=\frac{-1}{\sqrt{L^2_9+L^2_{10}+L^2_{11}}}
\end{equation} 
\begin{align*}
    x_c&\approx L^2(L_1L_9+L_2L_{10}+L_3L_{11})\\
    y_c&\approx L^2(L_5L_9+L_6L_{10}+L_7L_{11})\\
    c_x&\approx\sqrt{L^2(L^2_1+L^2_2+L^2_3)-x^2_c}\\
    c_y&\approx\sqrt{L^2(L^2_5+L^2_6+L^2_7)-y^2_c}\\
    p&\approx  (c_x+c_y)/2
\end{align*}
$$r_{11}\approx\frac{L(x_cL_9-L_1)}{c_x},r_{12}\approx\frac{L(y_cL_9-L_5)}{c_y},r_{13}\approx L\cdot L_9 $$
$$r_{21}\approx\frac{L(x_cL_{10}-L_2)}{c_x},r_{22}\approx\frac{L(y_cL_{10}-L_6)}{c_y},r_{23}\approx L\cdot L_{10} $$
$$r_{31}\approx\frac{L(x_cL_{11}-L_3)}{c_x},r_{32}\approx\frac{L(y_cL_{11}-L_7)}{c_y},r_{33}\approx L\cdot L_{11} $$
\begin{equation*}
    \begin{bmatrix}
        X_S\\Y_S\\Z_S
    \end{bmatrix}\approx -
    \begin{bmatrix}
        L_1&L_2&L_3\\
        L_5&L_6&L_7\\
        L_9&L_{10}&L_{11}\\
    \end{bmatrix}^{-1}\cdot 
    \begin{bmatrix}
        L_4\\L_8\\1
    \end{bmatrix}
\end{equation*}

\section{Tableau}

\begin{table}[H]
    \centering
    
    \begin{tabular}{c P{2.5cm} P{2.3cm} P{0.8cm} P{2.4cm}}
    \toprule
    \textbf{Classement} &\textbf{Code Git} & \textbf{Modèle IA} & \textbf{Année}  \\
    \midrule
    \small 1 & \small xuelunshen/gim &  \small GIM-RoMa & \small 2025  \\
    \small 2 & \small xuelunshen/gim &  \small GIM-DKM & \small 2024 \\
    \small 3 & \small parskatt/roma &  \small RoMa & \small 2023 \\
    \small 4 & \small  parskatt/dkm &  \small DKM & \small 2022\\
    \small 5 & \small xuelunshen/gim &  \small GIM-LoFTR & \small 2024 \\
    \small 6 & \small xuelunshen/gim &  \small GIM-LightGlue & \small 2024  \\
    \small 7 & \small zju3dv/LoFTR &  \small LoFTR & \small 2021 \\
    \small 8 & \small  ubc-vision/image... &  \small RootSIFT & \small 2012\\
    \small 9 & \small cvg/lightglue &  \small LightGlue & \small 2023  \\
    \small 10 & \small  magicleap/SuperGlue &  \small SuperGlue & \small 2019  \\
    \addlinespace

    \bottomrule
    \end{tabular}
    \caption{Classement des modèles IA de détermination de points homologues}
    \label{tabiaimagematching}
\end{table}


\begin{table}[H]
    \centering
{\tiny
        \renewcommand{\arraystretch}{2.5}
        \fontsize{10pt}{10pt}\selectfont
        \begin{tabular}{|>{\footnotesize}L{1.5cm}|>{\footnotesize}L{3cm}|>{\footnotesize}L{3cm}|}
            \hline
            \rowcolor{gray!30} % Couleur de fond de l'en-tête
            \textbf{Paires d'images} & \vphantom{g}\textbf{IA LightGlue} & \vphantom{g}\textbf{Algorithme de type SIFT de Agisoft}\\
            \hline
        \end{tabular}
        \renewcommand{\arraystretch}{1.5}
        \begin{tabular}{|>{\footnotesize}L{1.5cm}|>{\footnotesize}L{3cm}|>{\footnotesize}L{3cm}|}
            1 &Valide : 134 points \newline \underline{Invalide : 29 points}\newline \vspace{0.5em} Total : 163 points  &  Valide : 5 points \newline \underline{Invalide : 5 points}\newline  \vspace{0.5em} Total : 10 points \\
            \hline
            2 & Valide : 19 points \newline \underline{Invalide : 28 points}\newline \vspace{0.5em}  Total : 47 points & Valide : 0 points \newline \underline{Invalide : 0 points} \newline \vspace{0.5em}  Total : 0 points \\
            \hline
        \end{tabular}}
        \caption{Comparaison de la détection de points via l'IA LightGlue et via le logiciel Agisoft}
        \label{tabcompdetect}
\end{table}

\section{Figure}

\begin{figure}[H]
    \centering
    \includegraphics[width=5cm]{latex_logo.png}
    \caption[Logo latex]{Logo latex \footnotemark}
    \label{fig:latex}
\end{figure}
\sourcefootnotehref{https://www.latex-project.org/}

\section{Divers}
\begin{itemize}
    \item Orientation externe
    \begin{itemize}
        \item La position de l'image cible $S_{cible}=[X_{cible},Y_{cible},Z_{cible}]$
        \item L'orientation de l'image $R_\omega, R_\phi, R_\kappa$
    \end{itemize}
    \item Calibration interne (modèle \og Frame \fg de Agisoft)
    \begin{itemize}
        \item Le point principal d'autocollimation $PPA=[x_c,y_c]$
        \item La focale $f$
        \item Les coefficients de distorsions radiale ($k_{1-4}$), tangentielle ($p_{1-2}$) et affine ($b_{1-2}$)
    \end{itemize}
\end{itemize}

\begin{enumerate}
    \item Modèle photogrammétrique du lieu (calcul par aérotriangulation avec Agisoft)
    \item Recherche des points homologues entre une image utilisateur\footnote{Image acquise par un utilisateur dont les calibrations ne sont pas connues} et les images du modèle de base
    \item Détermination approximative des paramètres de calcul
    \begin{itemize}
        \item Détermination des coordonnées 3D des points homologues et filtre par reprojection
        \item Détermination de la position et de l'orientation approchée de l'image utilisateur avec la DLT et filtre des valeurs aberrantes avec les écarts aux observations
    \end{itemize}
    \item Détermination précise de l’orientation externe et de la calibration interne
    \begin{itemize}
        \item Calcul de la position de l'orientation et de la focale de l'image utilisateur par moindre carré non linéaire grâce à la détermination approximative et filtre des valeurs aberrantes avec les résidus normées.
        \item Calcul de la position de l'orientation, de la focale, du PPA et des distorsions par moindre carré non linéaire.
    \end{itemize} 
\end{enumerate}

\begin{itemize}
    \item un facteur de proximité spatiale, défini par la formule : \\
    $
\text{poids}_{\text{distance}} = 
\begin{cases}
1 - \dfrac{\text{dist}}{70}, & \text{si } \text{dist} < 70 \\
0.0000001, & \text{sinon}
\end{cases}
$\\ \\
\textit{\small La valeur déterminante de 70m a été choisie de manière empirique.}\\
\item un facteur d’orientation, calculé à l’aide du produit scalaire entre les vecteurs d’orientation (résultant en une valeur entre -1 et 1, soit le cosinus de l’angle).
\end{itemize}
\og Dès le XIX\up{e} siècle, les photographies retravaillées ont été utilisées dans des présentations pour illustrer l'impact d'un projet sur la vue urbaine et le paysage.\fg \cite{pousin:photomontage}(voir Figure \ref{fig:latex} à la page \pageref{fig:latex})