
\newcommand{\titlereport}{Modèle de base}
\newcommand{\studyname}{Résumé }
% \newcommand{\studyname}{Intégration automatique de futurs bâtiments dans une photographie}
\newcommand{\datedoc}{\today}
\newcommand{\datenumber}{{\number\day.\number\month.\number\year}}
\newcommand{\prof}{Adrien GRESSIN}
\newcommand{\descrVarEq}[2]{#1\textrm{ : }&\textrm{#2}}
\newcommand{\avec}{\textrm{\qquad avec}}
\newcommand{\donc}{\textrm{\qquad et donc}}
\newcommand{\figurecentre}[3][7]{
	\begin{figure}[H]
		\includegraphics[width=#1cm,center]{#2}
		\caption{#3}
	\end{figure}
}
\newcommand{\reminterne}[1]{{\color{red}\textbf{#1}}}
\newcommand{\bfit}[1]{\textbf{\textit{#1}}}

\usepackage{xparse}
\NewDocumentCommand{\paragraphtitle}{O{0.8em} m}{%
  \vspace{#1}%
  \hspace*{0.cm}%
  \textbf{\textit{\normalsize #2}}%
  \vspace{0.5em}%
}

\newcommand{\sourcefootnotehref}[1]{
	\footnotetext{source : \href{#1}{#1}}
}

\newcommand{\scu}{
	\begin{tikzpicture}[axis/.style={thick,->}]
		\coordinate (O) at (0, 0, 0);
		\draw[axis] (O) -- +(0.3, 0, 0);
		\draw[axis] (O) -- +(0, 0.3, 0);
		\draw[axis] (O) -- +(0, 0, 0.3);
	\end{tikzpicture}
}
\newcommand{\scuxyz}{
	\begin{tikzpicture}[axis/.style={thick,->}]
		\coordinate (O) at (0, 0, 0);
		\draw[axis] (O) -- +(1, 0, 0) node [right] {$X$};
		\draw[axis] (O) -- +(0, 1, 0) node [right] {$Y$};
		\draw[axis] (O) -- +(0, 0, 1) node [above] {$Z$};
	\end{tikzpicture}
}
\newcommand{\norm}[1]{\left\lVert \vec{#1} \right\rVert}
\newcommand\myemptypage{
		\null
		\thispagestyle{empty}
		\addtocounter{page}{-1}
		\newpage
	}
% Language settings
\usepackage[french]{babel}

%fontfamily
\usepackage[T1]{fontenc}
% \usepackage{lmodern}
% \fontfamily{cmr}\selectfont

\usepackage{helvet}
\renewcommand{\familydefault}{\sfdefault}


% \usepackage{fontspec}
% \setmainfont{Comic Sans MS}

% \usepackage{palatino}
\usepackage{fancyhdr}
\usepackage{lastpage}
\usepackage[export]{adjustbox}
\usepackage{graphicx}


\usepackage{ragged2e}  % Meilleure gestion des retours à la ligne

\usepackage{microtype}%gestion des césrure (retour à la ligne des grands mot avec un -)
% GESTION DES TABLEAUX
\usepackage{multicol}
\usepackage{multirow}
\usepackage{array}
% \usepackage{longtable}
\usepackage{esvect}
\graphicspath{ {./images/} }
\usepackage{geometry}
% \usepackage{mathtools}
\usepackage{amsmath}
% \usepackage{mathptmx}  % Utilise la police Times New Roman pour les mathématiques
% \everymath{\small}
% \usepackage[bitstream-charter]{mathdesign}
% \usepackage{subfigure}
\usepackage{caption}
\usepackage{tikz}
\usepackage{subcaption}
\usepackage{enumitem}
% \usepackage{amssymb}
\usepackage{float}
\usepackage{blindtext}
\setcounter{tocdepth}{2} %combien de niveau sont indiqué dans la table des matières
\usepackage[%
	hidelinks
]{hyperref} % referencing inside document (click on links) - has to be loaded as last !
\usepackage{footmisc} %note de base de page
%page vide
\usepackage{afterpage}
\usepackage{listings}
\usepackage{xcolor}
\usepackage{url}
\usepackage{titlesec}

\titleformat{\section}
  {\normalfont\LARGE\bfseries} % taille, police
  {\thesection}{1em}{\MakeUppercase}        % numérotation, espacement
\titleformat{\subsection}
  {\normalfont\Large\bfseries}
  {\thesubsection}{1em}{}

\titleformat{\subsubsection}
  {\normalfont\large\bfseries}
  {\thesubsubsection}{1em}{}

\usepackage{cleveref} % Paquet pour des références plus intelligentes
\usepackage{booktabs} %Tableau simplifié
\usepackage{moreverb} %vertbatimtabl
\usepackage{alltt}
%Bibliographie
% \usepackage{csquotes}
% % \usepackage[
% % 	backend=biber,
% % 	style=alphabetic,
% % 	sorting=ynt
% % 	]{biblatex}
% \usepackage[backend=biber,style=apa, giveninits=true,sorting=ynt, language=french]{biblatex}  
% \addbibresource{biblio.bib}
% \usepackage[numbers]{natbib}
\usepackage{cite}


\geometry{right=2cm, left=2cm, top=2cm, bottom=2.5cm} % Indique les dimensions des marges
\setlength\parindent{0pt} % Indique le décalage au début des paragraphes
\setlength{\parskip}{4pt} % Indique l'espace entre chaque "paragraphe"
\setlength{\arrayrulewidth}{0.2mm}
\setlength{\tabcolsep}{18pt}
\setlength{\headheight}{1.6pt}
\setcounter{MaxMatrixCols}{20}
\addtolength{\jot}{0.5em} % Espace entre les équations alignée en verticale
\setlist{before=\vspace{1pt},after=\vspace{2pt}} %Marge verticale avant et après avec les itemize
\setlist[itemize,1]{nosep, topsep=5pt, partopsep=2pt, itemsep=1.5pt, parsep=1.5pt, leftmargin=3em}
\setlist[itemize,2]{nosep, topsep=2pt, partopsep=2pt, itemsep=1.5pt, parsep=1.5pt, leftmargin=2em}
\setlist[enumerate,1]{nosep, topsep=5pt, partopsep=2pt, itemsep=1.5pt, parsep=1.5pt, leftmargin=3em}
\setlist[enumerate,2]{nosep, topsep=2pt, partopsep=2pt, itemsep=1.5pt, parsep=1.5pt, leftmargin=2em}




%paramètre tableau
\setlength{\tabcolsep}{18pt}
\renewcommand{\arraystretch}{1.5}
\newcommand{\head}[1]{%
\textcolor{white}{\textbf{#1}}}
\definecolor{mygraylight}{rgb}{.9,.9,.9}
% \rowcolors{2}{lightgray}{mygraylight}


%paramètre enumarte
\renewcommand{\labelenumii}{\arabic{enumi}.\arabic{enumii}}
\renewcommand{\labelenumiii}{\arabic{enumi}.\arabic{enumii}.\arabic{enumiii}}
\renewcommand{\labelenumiv}{\arabic{enumi}.\arabic{enumii}.\arabic{enumiii}}


% Informations du mémoire
\title{\titlereport}
\author{Bruno Della Casa Nom}
\date{\datedoc}


%====================== INFORMATION ET REGLES ======================

%rajouter les numérotation pour les \paragraphe et \subparagraphe
\setcounter{secnumdepth}{4}
\setcounter{tocdepth}{4}

\hypersetup{							% Information sur le document
pdfauthor = {Della Casa Bruno},			% Auteurs
pdftitle = {résumée},			% Titre du document
pdfsubject = {résumé},		% Sujet
pdfkeywords = {latex},	% Mots-clefs
pdfstartview={FitH}}					% ajuste la page à la largueur de l'écran
%pdfcreator = {MikTeX},% Logiciel qui a crée le document
%pdfproducer = {}} 